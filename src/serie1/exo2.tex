\begin{exo}
  \donnee{Sur le chemin de l’école, un étudiant de la HEIG–VD s’arrête toujours à la même station-service pour faire le plein d’essence. Il a constaté que les deux pompes de la station notées A et B ont la même probabilité d’être occupées. De plus, la probabilité que l’une des deux pompes au moins soit utilisée vaut 0.9 et celle que toutes les deux soient simultanément occupées est 0.5.}

  Pour faciliter la compréhension de cet exercice, on peut réaliser un diagramme de Venn.
  \begin{center}\includegraphics[scale=0.5]{ex2-diagrammeVenn}\end{center}
  Premièrement, posons que :
  \begin{multicols}{4}
    \begin{enumerate}[label=\alph*), parsep=0cm, itemsep=3mm, topsep=3mm]
      \item[ ] AO = La pompe A est occupée
      \item[ ] AL = La pompe A est libre
      \item[ ] BO = La pompe B est occupée
      \item[ ] BL = La pompe B est libre
    \end{enumerate}
  \end{multicols}
  \begin{center}$\Omega = \{(AL,BL),(AO,BL),(AL,BO),(AO,BO)\}$ \text{, évènements pas équiprobable !!!}\end{center}
  On sait que les probabilités que A soit libre sont les mêmes pour B. On note :
  \begin{center}$P(\{AL\}) = P(\{BL\})$\end{center}
  On sait aussi que :
  \begin{enumerate}
    \item[ ] $P(\text{au moins une pompe soite occupée}) = 0.9$
    \item[ ] $P(\text{les deux pompes soient occupées}) = 0.5$
    \item[ ] $P(\Omega) = 1$
  \end{enumerate}
  \begin{subexo}{Calculer la probabilité que les deux pompes soient disponibles.}
    Parmis les évènements probables on sait que :
    \begin{center}$P\{(AO,BL),(AL,BO),(AO,BO)\} = 0.9$\end{center}
    Puisque nous cherchons la probabilité du couple manquant $(P\{(AL,BL)\})$, on pose :
    \begin{center}$P(\Omega) = P\{(AL,BL)\} + P\{(AO,BL),(AL,BO),(AO,BO)\}$\end{center}
    En remplaçant les probabilités connues par leurs valeurs, on obtient :
    \begin{center}$1 = P\{(AL,BL)\} + 0.9$\end{center}
    On peut donc en conclure que $P\{(AL,BL)\}  = 0.1$ et donc que la probabilité que la pompe A et B soient libre est de $0.1$
  \end{subexo}
\begin{subexo}{Déterminer la probabilité que la pompe A soit libre.}
  Rappel : $P(A\cup B) = P(A) + P(B) - P(A \cap B)$ \\ \\
  Premièrement, on pose la formule énoncée dans le rappel en remplaçant avec nos valeurs :
  \begin{center}$P(AL\cup BL) = P(AL) + P(BL) - P(AL \cap BL)$\end{center}
  Maintenant, essayons d'exprimer $P(AL\cup BL)$ avec ce que nous connaissons : \\
  Avec la Loi de De Morgan on a :
  \begin{center}$P(\overline{AL\cup BL}) = P(\overline{AL} \cap \overline{BL}) = P(AO \cap BO)$\end{center}
  On sait que $P\{(AO,BO)\}=0.5$ donc que $P(AO \cap BO)=0.5$, on peut donc poser :
  \begin{center}$P(\overline{AL\cup BL}) = 1 - P(AL \cup BL)$\end{center}
  À présent on sait que :
  \begin{enumerate}
    \item[ ] $P(AL \cup BL) = 0.5$
    \item[ ] $P(AL \cap BL) = 0.1$ , voir exercice $a)$
  \end{enumerate}
  On peut donc compléter la formule $P(AL\cup BL) = P(AL) + P(BL) - P(AL \cap BL)$ avec les valeurs :
  \begin{center}
    $0.5 = P(AL) + P(BL) - 0.1$
  \end{center}
  Comme ennoncé en début d'exercice, on a $P(AL) = P(BL)$. On peut donc poser :
  \begin{center}
    $2P(AL) = 0.6$ \\ $P(AL) = 0.3$
  \end{center}
  On peut donc conclure que les probabilités que la pompe A soie libre sont de 0.3
\end{subexo}
\begin{subexo}{Calculer la probabilité que la pompe A soit occupée mais la pompe B disponible}
  Posons simplement la somme des probabilités que A soit occupée
  \begin{center}
    $P(AO) = P(AO \cap AO) + P(AO \cap BL)$
  \end{center}
  Sachant que $P(AL) = 0.3$ , on déduit que $P(AO) = 0.7$. On peut donc retourner la formule au-dessus pour obtenir :
  \begin{center}
    $P(AO \cap BL) = P(AO) - P(AO \cap AO)$
  \end{center}
En remplaçant par les valeurs connues on a :
\begin{center}
  $P(AO \cap BL) = 0.7 - 0.5 = 0.2$
\end{center}
On peut donc dire que la probabilité que la pompe A soit occupée et que la pompe B disponible est de $0.2$
\end{subexo}
\end{exo}
