\begin{exo}
  \donnee{Tous les jours de la semaine, un enseignant se rend à la même heure à l'école où il enseigne en suivant
  systématiquement le même chemin. Sur son trajet se trouvent deux carrefours où la circulation est réglée
  par un feu de signalisation ; rouge ou orange, pas d’autres signaux lumineux. La probabilité que le feu du
  premier carrefour soit au orange lors du passage de l’enseignant vaut p et celle du second 2p. De plus, la
  probabilité qu’au moins un des deux feux soit orange lors du passage de l’enseignant est 0.28. On suppose
  que les feux de signalisation des deux carrefours fonctionnent indépendamment l’un de l’autre.
  Considérons les  événements :}
  $$A : \text{“Le feu du premier carrefour est orange” et } B : \text{“Le feu du second carrefour est orange”.}$$
  \begin{subexo}{Traduire l'énoncé en langage probabiliste à l'aide des événements énoncés ci-dessus}

    La probabilité que le feu du premier carrefour soit au orange lors du passage de l'enseignant vaut p.
    $$P(A) = p$$
    La probabilité que le feu du deuxième carrefour soit au orange lors du passage de l'enseignant vaut 2p.
    $$P(B) = 2p$$
    la
    probabilité qu’au moins un des deux feux soit orange lors du passage de l’enseignant est 0.28
    $$P(A\cup B) = 0.28$$

  \end{subexo}
  \begin{subexo}{Calculer la probabilité p}
    Puisque les événements sont indépendants, nous avons :
        \begin{align}
          P(A \cap B) &= P(A) P(B)\\
      &= p \cdot 2p\\
      &= 2p^2
        \end{align}
      Nous savons aussi que
        \begin{equation}
          P(A \cup B) = P(A) + P(B) -P(A \cap B)
        \end{equation}
        L'équation (1) est vraie uniquement si les événements sont indépendants. Tandis que la (4) est toujours vrai !!
        \begin{align}
          P(A \cup B) &= P(A) + P(B) -P(A \cap B)\\
           &= 0.28\\
           &= p + 2p - 2p^2
        \end{align}
        Nous devons donc trouver les solutions de l'équation quadratique suivante:
        $$2p^2 - 3p + 0.28 = 0$$
          Nous avons : $\Delta = b^2 -4ac = 9 -4\cdot 2\cdot 0.28$\newline
           et $\sqrt{\Delta} = 2.6$
           $$\delta^+ = \frac{-(-3) + 2.6}{2\cdot2} = 1.4$$
           $$\delta^- = \frac{-(-3) - 2.6}{2\cdot2} = 0.1$$
          $p$ étant une probabilité, son domaine de définition est : $\mathbb{D} = [0;1]$ et $\delta^+ \notin \mathbb{D}$
          Donc $p = 0.1$
  \end{subexo}
\end{exo}
