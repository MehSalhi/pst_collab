\begin{exo}
  \donnee{On jette de suite deux dés équilibrés. Considérons les événements :
    \\• A : “la somme des faces vaut 6”;
    \\• B : “le premier dé donne 4”;
    \\• C : “la somme des faces vaut 7”.
    \\Les événements A et B sont-ils indépendants ? Qu’en est-il des événements B et C ? Donnez une interprétation intuitive des résultats.}
  \begin{subexo}{Les événements A et B sont-ils indépendants ? Qu’en est-il des événements B et C ? Donnez une interprétation intuitive des résultats.}
    \begin{flushleft}
      L'univers $\Omega = \{(1,1),...,(6,6)\}$ Pour répondre savoir si un événement $A$ et $B$ sont indépendant, on peut utiliser la définition: $A$, $B$ indépendant $\Leftrightarrow P(A | B) = \dfrac{P(A \cap B)}{P(B)}=P(A)$ Commençons par calculer les probabilités:
      \\$P(A) = \dfrac{5}{36}$ car sur 36 jets possible il y a $\{(1,5),(2,4),(3,3),(4,2),(5,1)\}$
      \\$P(B) = \dfrac{1}{6}$
      \\$P(C) = \dfrac{1}{6}$ car sur 36 jets possibles il y a $\{(1,6),(2,5),(3,4),(4,3),(2,5),(6,1)\}$
      \\$P(A \cap B) = \dfrac{1}{36}$ c'est à dire le cas $\{(4,2)\}$
      \\Vérifions l'indépendance de A et B: $P(A | B) = \dfrac{\frac{1}{36}}{\frac{1}{6}}=\dfrac{1}{6} \neq P(A)$ Donc A et B sont dépendants.
      \\ Procédons à la suite:
      \\$P(B\cap C) = \dfrac{1}{36}$ c'est à dire le cas $\{(4,3)\}$
      \\Vérifions l'indépendance de B et C: $P(B | C) = \dfrac{\frac{1}{36}}{\frac{1}{6}}=\dfrac{1}{6} = P(C)$ Donc B et C sont indépendants
    \end{flushleft}
  \end{subexo}
\end{exo}
