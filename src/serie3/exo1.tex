\begin{exo}
  \donnee{Une boîte contient deux billes noires et trois billes blanches. On tire une bille à la fois, sans remise,
    jusqu’à ce que la première bille noire apparaisse. Dès que la bille noire est tirée, aucune autre bille est extraite de la boîte. Désignons par $X$ le nombre de tirage nécessaires. En utilisant un arbre binaire,}
    \begin{subexo}{Déterminer  l'ensemble fondamental $\Omega$ de l'expérience aléatoire à l'aide des événements :}
        \begin{itemize}
          \item $N$ Une bille noire est tirée
          \item $B$ Une bille blanche est tirée
        \end{itemize}
        \begin{center}
        \begin{tikzpicture}[grow=right]
          \node[]{Arbre des choix}
          child{
            node[]{$N$}
            edge from parent
            node[below]{$\frac{2}{5}$}
          }
          child{
            node[]{$B$}
            child[]{
              node[]{$N$}
              edge from parent
              node[below]{$\frac{2}{4}$}
            }
            child[]{
              node[]{$B$}
              child[]{
                node[]{$N$}
                edge from parent
                node[below]{$\frac{2}{3}$}
              }              
              child[]{
                node[]{$B$}
                child[]{
                  node[]{$N$}
                  edge from parent
                  node[above]{$\frac{1}{1}$}
                }
                edge from parent
                node[above]{$\frac{1}{3}$}
              }
              edge from parent
              node[above]{$\frac{2}{4}$}
            }
          edge from parent
          node[above]{$\frac{3}{5}$}
          };
        \end{tikzpicture}
        \end{center}
        Nous arrivons à l'arbre si dessus. En effet avec cet arbre, nous avons toutes les possibilités que peut prendre $X$
        la probabilité est notée sur chaque branche.
        Ainsi lors du premier tirage, nous avons intuitivement $\frac{3}{5}$ "chances" de tirer une boule blanche, et nous avons une probabilité de tiré une boule noir égal à $\frac{2}{5}$ 
    \end{subexo}
    \begin{subexo}{Calculer la loi de probabilité de $X$}
      \begin{center}
        
        \begin{tabular}{p{3cm}|p{2cm}p{2cm}p{2cm}p{2cm}p{2cm}}
          \midrule
          X = xi  & 1   & 2         & 3             & 4                   \\ \toprule
          P(X=xi) & $\frac{2}{5}$    & $\frac{3}{5}\cdot \frac{1}{2}$ & $\frac{3}{5} \cdot \frac{1}{2}  \cdot\frac{2}{3} $& $\frac{3}{5} \cdot \frac{1}{2}  \cdot \frac{1}{3}\cdot \frac{1}{1}$ \\ \midrule
          \end{tabular}
      \end{center}
      On suit simplement la solution à travers l'arbre en multipliant les probabilités
    \end{subexo}
\end{exo}
