\begin{exo}
  \donnee{Imaginons qu’on souhaite transmettre les réalisations $x_1, x_2, . . . , x_n$ d’une variable aléatoire discrète X d’un point d’observation A à un point de réception B à l’aide d’un canal de communication ne pouvant transférer que des 0 ou des 1. Ainsi, les valeurs prises par X devront être codées en chaînes formées uniquement de 0 et de 1 avant d’être transmises. Pour éviter toute ambiguïté, on exige qu’un code ne puisse pas être une extension d’un autre.
  \\Comme exemple, supposons que les réalisations de X sont $x_1, x_2, x_3, x_4$. Comme code possible, on peut envisager
  \\$$x_1 \leftrightarrow 00, x_2 \leftrightarrow 01, x_3 \leftrightarrow 10, x_4 \leftrightarrow 11$$Ainsi, si X prend la valeur $x_1$, le message envoyé en B sera 00, il vaudra 01 si X = $x_2$ et ainsi de suite. Un autre code possible est
  \\$$x_1 \leftrightarrow 0, x_2 \leftrightarrow 10, x_3 \leftrightarrow 110, x_4 \leftrightarrow 111$$Un objectif du codage consiste tout naturellement à minimiser le nombre espéré de bits nécessaires pour transmettre l’information. Ainsi, un code est dit plus efficace qu’un autre si son nombre espéré de bits est plus petit que celui nécessaire à l’autre code.}
\end{exo}
