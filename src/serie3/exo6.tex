\begin{exo}
  \donnee{La distribution de Benford prétend que pour des mesures d’une certaine
    quantité physique, la probabilité de tomber sur un 1 est de l’ordre de 30 \%
    alors que celle de tomber sur un 9 est de l’ordre de 4.6 \%.
    Cette distribution n’est applicable que pour un grand nombre d’objets mesurables.
    Elle s’applique par exemple aux longueurs de tous les fleuves du monde,
    aux prix des articles d’un supermarché ou encore aux volumes des tables de
    logarithmes d’une encyclopédie. En effet, les premières pages de chaque volume
    sont toujours plus usées que les suivantes, celles du premier volume étant
    plus utilisées que celles du second et ainsi de suite. Il en sera de même
    pour les nombres d’une comptabilité. Ainsi, pour détecter d’éventuelles
    fraudes dans les déclarations d’impôts de grandes entreprises, le fisc américain
    a utilisé la distribution de Benford. En effet, dans une comptabilité
    de grande envergure, qui bien souvent est formée de nombreuses pages de
    chiffres, les comptables essayaient pour déjouer le fisc de créer des
    données truquées en tirant des nombres au hasard. L’idée n'était pas
    très bonne. En se livrant à une petite analyse statistique à l’aide
    de la distribution de Benford, il était possible d’identifier des
    fraudes dans les déclarations d’impôts de grandes sociétés.
    La distribution de probabilités de Benford se trouve dans la table ci-dessous.
    \begin{center}
      \begin{tabular}{llllllllll}
        \toprule
        valeur      & 1     & 2     & 3     & 4     & 5     & 6     & 7     & 8     & 9     \\
        \midrule
        probabilité & 0.301 & 0.176 & 0.125 & 0.097 & 0.079 & 0.067 & 0.058 & 0.051 & 0.046 \\
        \bottomrule
      \end{tabular}
    \end{center}
  }
  \begin{subexo}{calculer l'espérance mathématique
      d'une variable aléatoire X issue d'une distribution de Benford}
    L'espérance est la moyenne pondérée càd définie comme
    \[
      \mathbb{E}(X) = \sum_{i=1}^n x_i \cdot p_i
    \]
    avec $p_i$ la probabilité que la variable aléatoire prenne la valeur $x_i$
    \begin{align*}
      \mathbb{E}(X) & = \sum_{i=1}^n x_i \cdot p_i                                                                                                               \\
                    & = 1 \cdot 0.301 + 2\cdot 0.176 + 3\cdot 0.125+ 4 \cdot 0.097 + 5 \cdot 0.079 + 
                    6 \cdot 0.067 + 7 \cdot 0.058+ 8 \cdot 0.051 + 9\cdot 0.046 \\
                    & = 3.441
    \end{align*}
  \end{subexo}
  \begin{subexo}{Determiner la probabilité que $P(X < 7 | X \ge 4)$}
    Nous pouvons réécrire la probabilité conditionel de la manière suivante

    \begin{align*}
      P(X < 7 | X \ge 4) =& \frac{P(X < 7 \cap X \ge 4)}{P(X \ge 4)} \\\\
      =& \frac{0.097 + 0.079 + 0.067}{0.097 +0.079 + 0.067 + 0.058 + 0.051 +  0.046 }
    \end{align*}
  \end{subexo}
\end{exo}
