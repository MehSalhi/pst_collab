\begin{exo}
  \donnee{Une petite chenille descend doucement, lentement le long du grillage
    représenté dans la Figure 1. À chaque point gras de la figure appelé épissure,
    elle choisit la maille à votre gauche une fois sur trois, la maille à votre
    droite deux fois sur trois. La chenille descend quatre niveaux.}
  \begin{subexo}{Déterminer la loi de probabilité de X i.e. les probabilités
      $P(X = x)$ avec $x = 0, 1, 2, 3, 4$.}
    Nous voyons qu'à chaque épissure nous avons deux possibilités.
    L'une avec $\frac{2}{3}$ de chance d'être pris et l'autre avec 1 tier.
    Ainsi pour partir du départ et arriver à 0,
    nous devons toujours prendre le chemin de droite donc la réalisation 0 arrive avec une
    probabilité $(\frac{2}{3})^3 $ nous voyons la même chose avec la réalisation 4 où il faut prendre
    à chaque fois choisir le chemin de gauche qui arrive avec une proba de $\frac{1}{3}$ indépendamment
    des choix précédents.
    Nous pouvons faire la même chose pour la $X = 1$. En effet il pour avoir cette réalisation de X,
    il nous faut prendre 3 fois à droite et 1 fois à gauche. Cependant il faut noter
    que la position de notre gauche peut être n'importe où dans le chemin.
    Nous pouvons faire dddg ou ddgd ou dgdd ou gddd (d = droite et g = gauche)
    et nous arrivons avec chaque chemin à la réalisation 1. Nous avons donc
    4 possibilités. Pour $x = 2$ nous avons 2 gauches et 2 droites, nous devons donc dans notre chemin De
    4 étapes placé nos deux gauches. il existe 6 possibilités de placer 2 objets parmis 4.
    Nous comprenons maintenant qu'il s'agit d'une variable aléatoire discrète provenant d'une
    \textbf{distribution binomiale}. Nous avons donc la lois de probabilité suivante :
    \[\begin{tabular}{@{}l|lllll@{}}
        X = x  & 0                                 & 1
               & 2                                 & 3
               & 4                                                                     \\ \midrule
        P(X=x) & $(\frac{2}{3})^4$                 & $4(\frac{2}{3})(\frac{1}{3})$
               & $6(\frac{2}{3})^2(\frac{1}{3})^2$ & $4(\frac{2}{3})^1(\frac{1}{3})^3$
               & $(\frac{1}{3})^4$
      \end{tabular}
    \]
  \end{subexo}
  \begin{subexo}{De quelle distribution est issue la variable aléatoire X ? 
    Préciser ses paramètres.}
  Comme dit précédement il s'agit d'une lois d'une\textbf{ distribution binomiale}
  pour le voir il faut noter :
  \begin{enumerate}[]
    \item le fait d'avoir uniquement deux possibilités à chaque moment de \textit{"choix"}
    \item la symétrie dans les probabilités 
  \end{enumerate}
  \end{subexo}
\end{exo}
