\begin{exo}
  \donnee{Un signal transmis par un canal est reçu sans erreur avec une probabilité de 0.9.
    On suppose que les transmissions sont indépendantes les une des autres}
  \begin{subexo}{Calculer la probabilité que parmi les 20 signaux transmis, plus de deux
      signaux soient reçus avec erreur }
    En se rappelant la théorie, on sait que la loi binomiale compte le nombre de succès
    parmis n épreuves indépendantes.
    Comme dans les séries précédentes, on voit "plus de/que" il faut donc faire
    1 - le complémentaire de nos évènements puisque toutes nos lois permettent de calculer
    $P(X < x_i)$
    Appelons A l'évenement "avoir plus de deux erreurs".
    Plus explicitement nous devons calculer :
    \begin{align*}
      1 & - P("\text{avoir exactement 0 erreur}")  \\
        & - P("\text{avoir exactement 1 erreur"})  \\
        & - P("\text{avoir exactement 2 erreurs}") \\
      \\
        & = P(A)
    \end{align*}
    \[
      P(A) = 1 - \binom{20}{0} \cdot 0.9^{20}(0.1)^0
      - \binom{20}{1}(0.9)^{19}(0.1)^1
      - \binom{20}{2}(0.9)^{18}(0.1)^{2}
      \\ = 0.32
    \]
  \end{subexo}
  \begin{subexo}{Déterminer la probabilité que le cinquième signal transmis soit le premier signal reçu avec une erreur}
    Il s'agit de la probabilité que le premier soit reçu sans erreur ET que le second le soit aussi $\cdots$ ET que le 5iem soit recu AVEC erreur
    Donc, comme les événements sont indépendants nous pouvons multiplier les probabilités.
    Nous n'avons pas de coefficient devant puisque nous avons un seul arrangement possible
    pour avoir 4 justes suivit d'un faux
    \[
      \underbrace{0.9 \cdot 0.9 \cdot 0.9 \cdot 0.9 \cdot}_{\text{proba que les 4 premiers soit justes}} 0.1
      \\ = 0.06561
    \]

  \end{subexo}
  \begin{subexo}{Calculer la probabilité que le dixième signal
      transmis soit le quatrième signal re4u avec erreur.}
    Nous avons donc 9 signaux transmis dont 3 avec erreurs suivit d'un signal avec une erreur
    \[
      \underbrace{\binom{9}{3}(0.9)^6 \cdot (0.1)^3}_{\text{9 signaux dont 3 avec erreurs}} 
      \cdot \overbrace{0.1}^{\text{une erreur}}
    \]
    la proba vaut donc 0.004464
  \end{subexo}
\end{exo}
