\begin{exo}
  \donnee{Une platforme petrolière a été construite à une hauteur de 8 mètres au 
  dessus du niveau
  de la mer. Selon les statistiques, les vagues atteignent le plateau de la plate-forme 
  une année donnée avec probabilité 0.05. On admet que les vagues parviennent au plateau 
  indépendamment des années}
  \begin{subexo}{Déterminer la probabilité qu'a partir de cette année les vagues 
    déferleront sur la plate-forme pour la première fois dans l'une des 
    cinq prochaines années}
    On se rappelle la théorie sur la loie géométrique et on se rappelle qu'elle compte le nombre
    d'essaie necessaire avant d'avoir la première épreuve réussite.
    Nous avons comme dans l'exercice précédent une suite d'épreuve indépendante avec une probabilité
    1-p d'échouée. Définission p la probabilité que les vagues atteignent la platforme.
    nous avons la probabilité que "les vagues atteignent la platforme dans les 5 prochaines années"
    qui est égale à la proba qu'elle les atteignent l'année prochaine, ou qu'elle ne réussisent pas la première année, 
    mais y arrivent l'année suivante, ou durant la troisième année(mais pas durant les deux premières)
    etc.
    Donc la proba vaut 
    \[
    0.05^1 + 0.95\cdot0.05 + 0.95^2\cdot0.05  + 0.95^3\cdot0.05  + 0.95^4\cdot0.05  
    \]
    Donc la proba = 0.226219
  \end{subexo}
  \begin{subexo}{En sachant que les vavues ne parviendront pas au plateau au minimum pendant les 
    cinq prochaines années, cacluler la probabilité qu'elles déferleront sur la platforme pour la première fois exactement dans huit ans}
    Les vagues n'ont pas de mémoire, donc la proba ne change pas en sachant que les 5 prochaines années les vagues ne dépasseront pas le niveau.
    Ca revient à calculer la proba que durant les 3 prochaines années les vagues ne dépasseront pas le niveau
    \[0.95^2 \cdot 0.05 = 0.045125\]
    
  \end{subexo}
\end{exo}
