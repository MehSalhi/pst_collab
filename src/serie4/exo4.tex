\begin{exo}
  \donnee{Un serveur de calcul reçoit des requêtes à fréquence régulière.
    Elles sont traitées indépendamment les
    unes aprés les autres dans leur ordre d’arrivée. On suppose que le nombre de
    requêtes par minute qui
    arrivent au serveur peut être modélisé par un processus de Poisson.
    Les requêtes sont envoyées vers le
    serveur à un rythme moyen de 6 requêtes par minute.}

  \begin{subexo}{Déterminer la probabilité qu'en l'espace d'une minute et demi,
      au moins deux reqêtes arrivent au server.}
    Nous sommes dans le cas d'un processus de poisson. Donc
    \[
      P(N(t) = x) = e^{-\lambda t} \cdot \frac{(\lambda t)^x}{x!}
    \]
    Où
    \begin{enumerate}
      \item $N(t)$ compte le nombre d'occurence
      \item $\lambda$ est la fréquence d'occurence de l'événement observé = 6
      \item $t$ est le temps durant lequel on compte le nombre d'occurence  = 1.5 minutes
    \end{enumerate}
    Au vu de l'énoncé "au moins" nous savons que nous devons faire le complémentaire puisque nos
    lois de probabilité nous informent toujours sur $P(X < x)$.
    \begin{align*}
      P(N(1.5) \ge 2) = & 1 - \overline{P(N(1.5) \ge 2)} \\
      =                 & 1 - P(N(1.5) < 2)
    \end{align*}
    Comme nous sommes dans un cas discret, nous avons :
    \[
      P(N(1.5) < 2) = P(N(1.5) = 0) + P(N(1.5) = 1)
    \]
    et donc
    \begin{align*}
      P(N(1.5) \ge 2) = & 1 - P(N(1.5) = 0) - P(N(1.5) = 1)                                                                         \\
      =                 & 1 - e^{-6 \cdot 1.5} \cdot \frac{(6 \cdot 1.5)^0}{0!} - e^{-6 \cdot 1.5 } \cdot \frac{(6\cdot 1.5)^1}{1!} \\
      =                 & 1- e^{-6 \cdot 1.5} - 9 \cdot e^{-6 \cdot 1.5}                                                            \\
      =                 & 1- 10e^{-9}
    \end{align*}
  \end{subexo}
  \begin{subexo}{En sachant que les deux requêtes sont arrivées en l'espace d'une minute, calculer
      la probabilité qu'elles soient toutes les deux arrivées dans les 20 premières secondes}

    Nous voyons ici le fameux "en sachant" qui nous dit d'utiliser bayes.
    \[
      P\biggl( N(t_1) = 2 \biggl| N(t_2) = 2\biggr)
    \]
    \begin{enumerate}
      \item $t_1$ = 20 secondes = $\frac{1}{3}$ minutes
      \item $t_2$ = 1 minute
    \end{enumerate}
    Nous mettons tout en minutes pour avoir des unités cohérente par rapport à notre $\lambda$ qui est en minute
    \begin{align*}
      P\biggl( N(1/3) = 2 \biggl| N(1) = 2\biggr)
      = & P\biggl( e^{-6 \cdot 1/3} \cdot \frac{(6 \cdot 1/3)^2}{2 !} \biggl|
      e^{-6 \cdot 1} \cdot \frac{(6 \cdot 1)^2}{2 !}  \biggr)
    \end{align*}
    $$P(N(1/3) = 2) = e^{-6 \cdot 1/3} \cdot \frac{(6 \cdot 1/3)^2}{2 !}$$
    Et
    $$
      P(N(1) = 2) = e^{-6 \cdot 1} \cdot \frac{(6 \cdot 1)^2}{2 !}
    $$
    Nous savons que $N(t_1)$ est compris dans $N(t_2)$ puisque si les requêtes arrivent dans les 20 premières secondes, elles est forcément dans la première minute.
    Donc
    \begin{align*}
      P\biggl( N(1/3) = 2 \biggl| N(1) = 2\biggr) = & \frac{N(1/3) = 2 \cap N(1) = 2 }{N(1) = 2}                                                                   \\
      =                                             & \frac{ e^{-6 \cdot 1/3} \cdot \frac{(6 \cdot 1/3)^2}{2 !}}{ e^{-6 \cdot 1} \cdot \frac{(6 \cdot 1)^2}{2 !} }
    \end{align*}
    Et ainsi
    \[
      P\biggl( N(t_1) = 2 \biggl| N(t_2) = 2\biggr) = \frac{e^{-2}\cdot 2}{e^{-6}\cdot 13}
    \]


  \end{subexo}
\end{exo}
