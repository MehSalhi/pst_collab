
\begin{exo}
  \donnee{Le système informatique d’une petite PME est formé de deux serveurs indépendants. On suppose que la
  	durée de vie (temps écoulé avant un crash) de chacun d’eux peut être modélisée par une variable aléatoire
  	issue d’une distribution exponentielle. La durée de vie espérée de chacun des deux serveurs est de 150
  	jours.}
 
 \begin{subexo}{Calculer la probabilité que l’un des deux serveurs indépendamment de l’autre fonctionne pendant au moins 100 jours.}
 	Formule de répartition d'une distribution exponentielle (formulaire):
 		\begin{center}
 		$ F_x(x) =\begin{cases}
 		0 & \text{si $x < 0$} \\
 		1 - e^{-\lambda x} & \text{si $x \geq 0$} \\
 		\end{cases}$
 		\end{center}
 	La durée de vie espérée est de 150 jours (donnée), donc: 
 		\begin{center}
 			$\mathbb{E} = 150$ \\ 
 			et 	pour une distribution exponentielle, l'espérance vaut \\ 
 			$\mathbb{E} = \frac{1}{\lambda}$ donc $\lambda = \frac{1}{150}$ donc, 
 			
 			\begin{align*}
 			P(x > 100) &= 1 - P(x \le 100) \\
 			&= 1 - (1 - e^{-\frac{1}{\lambda}x})  \\
 			&= 1 - (1 - e^{-\frac{1}{150}100})  \\
 			&= 1 - (1 - e^{-\frac{100}{150}})  \\
 			&= 1 - (1 - e^{-\frac{10}{15}})  \\
 			&= e^{-\frac{2}{3}}  \\
 			\end{align*}
 		\end{center}
 \end{subexo}
  \begin{subexo}{Déterminer la probabilité que les deux serveurs fonctionnent ensemble et simultanément pendant au moins 100 jours.}
  	On cherche P($x_1 \geq 100$ et $x_2  \geq 100$) \\
  	Comme les deux serveurs sont indépendants, $f_{x_1,x_2} = f_{x_1}(u)*f_{x_2}(u)$ et donc :\\
		\begin{center}
			$e^{-2/3} * e^{-2/3} = e^{4/3}$
		\end{center}
  \end{subexo}
\end{exo}
